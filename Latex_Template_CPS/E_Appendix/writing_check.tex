\section{General Checks}
\begin{itemize}

\item Do you summarize your contribution in the introduction?
\item Is the bibliography consistent? (If you abbreviate first names once, do it all the way through. If you have page numbers once, have page numbers throughout.)
\item Is the spelling of all proper names correct? You would hate to get your paper reviewed by someone who would find his name misspelt in your paper.
\item Are the captions correct? Do you put the table caption before or after the table? Do you put the figure caption before or after the figure? Do you center captions or not?
\item Do you refer to a figure as "Fig. 1" or as "Figure 1"? Which one is correct?
\item Are all internal references correct? If you refer to Fig. 10, does Figure 10 exists? (Some LaTeX package can mess this up, so always check!) Are all tables and figures referenced in the text?
\item If this is a recurring conference or a journal, have you compared your paper with ten or so other articles to make sure that yours is consistent with how these other papers look and feel? For example, if all published articles use 10 pages for the introduction, make sure you do too.
\item Do you use the right fonts? Be watchful: sometimes the font for the section header can differ from the font used in the main text.
\item Avoid unnecessary lines and borders. Tools like Excel tend to put black borders around figures: get rid of it.
\item It is a great idea to use color where appropriate, especially when you expect your readers to read the electronic version of your document. However, you should not use color unnecessarily: tools like Microsoft Word tend to put all hyperlinks in blue, is this really necessary? Moreover, you should make sure that people can print your paper (in black ink) and still understand the content.
\item If you have collected data or written software, have you tried making it available online?
\item Do you know your story-line? Before you start writing, write your story-line on a napkin!
\item Try to build up your story like a funnel: start with a broad introduction that makes sure EVERYONE understands in which field you are and what you're talking about. THEN zoom in onto your problem and go into the details.
\item Is your Table of Contents well-balanced?
\item Be sufficiently repetitive: say what you are gonna say, say it, say what you have said.
\item Try to structure your paper such that sections have the same length. Use proper structure (at least introduction, motivation, and conclusion should be present)
\item Did you draw the bigger picture (e.g in introduction, conclusion, discussion)?
\item Is the introduction \& conclusion understandable for a beginner in the field? Have you explained all basic concepts?
\item In Theses: Have good future work sections. In papers, do not go overboard with future work sections.
\item Have someone proof read your thesis -- also and maybe especially for style issues!
\item Is the layout of each page elegant?
\item Did you spell check? (Really! Please do it!)
\item Can you replace some mathematical notation by plain English?
\item Are all terms defined?
\item Is the mathematical notation consistent? (If you use t for time in the first section, do you use t to note the term in the second section?)
\item Are names consistent? If you called an algorithm Bozo3 in the introduction, don’t call it BOZO-3 in the conclusion.
\item Do the title and the abstract invite the reader to read the rest of the paper?
\item Do you summarize your contribution in the introduction?
\item Is the bibliography consistent? (If you abbreviate first names once, do it all the way through. If you have page numbers once, have page numbers throughout.)
\item Is the spelling of all proper names correct? You would hate to get your paper reviewed by someone who would find his name misspelt in your paper.
\item Are the captions correct? Do you put the table caption before or after the table? Do you put the figure caption before or after the figure? Do you center captions or not?
\item Do you refer to a figure as "Fig. 1" or as "Figure 1"? Which one is correct?
\item Are all internal references correct? If you refer to Fig. 10, does Figure 10 exists? (Some LaTeX package can mess this up, so always check!)
\item Are all tables and figures referenced in the text?
\item If this is a recurring conference or a journal, have you compared your paper with ten or so other articles to make sure that yours is consistent with how these other papers look and feel? For example, if all published articles use 10 pages for the introduction, make sure you do too.
\item Do you use the right fonts? Be watchful: sometimes the font for the section header can differ from the font used in the main text.
\item Avoid unnecessary lines and borders. Tools like Excel tend to put black borders around figures: get rid of it.
\item It is a great idea to use color where appropriate, especially when you expect your readers to read the electronic version of your document.
\item However, you should not use color unnecessarily: tools like Microsoft Word tend to put all hyperlinks in blue, is this really necessary? Moreover, you should make sure that people can print your paper (in black ink) and still understand the content.
\item If you have collected data or written software, have you tried making it available online?
\item Does it say clearly what our contribution is? Reviewers are lazy, they do not want to have to figure out what your message is. Spend some time telling us exactly what your contribution is. Spell it out, do not assume we will read the paper carefully.
\item Do I have original examples and original data sets?
\item Do I have sufficient experimental evidence?'' You need to confront your idea with the real world and report on how well it fares. Compare explicitly your results with the best results elsewhere.
\item Does it contain weak unnecessary results? If you derived ten theorems but only one is necessary, throw the rest of them in your drawers. We do not want to know about useless results!
\item Is it marred by the technical details? Technical papers made of several small ideas are usually uninteresting. Similarly, we are not interested how many seconds it took on your i386 computer from 1990s or longer ago.
\item Do I have a picture summarizing the approach? Really, even if you feel silly doing it or that you think you can’t draw. A picture can help tremendously in communicating difficult ideas.
\item A sexy start: tell the reader early why he should read your paper. Don’t summarize, sell! A good abstract tells us why we should read this paper, it does not summarize the paper. Convince us early that your paper is important. For example, the Kent Beck recipe for a good 4-sentence abstract is: (1) state the problem (2) say why it is interesting (3) say what your solution achieves (4) say what follows from your solution.
\item You should clearly say what your contribution is. Reviewers are lazy, they do not want to have to figure out what your message is. Spend some time telling us exactly what your contribution is. Spell it out, do not assume we will read the paper carefully.
\item A review of related work in the introduction: you can relate your own contribution to all of the related work.
\item A large reference section: people like to be cited, so make sure you cite every paper that might have some relevance.
\item Experimental evidence: you need to confront your idea with the real world and report on how well it fares. Compare explicitly your results with the best results elsewhere.
\item Acknowledge the limitations of your work.
\item Relevant and non-obvious theoretical results: it is easier for people to build on your work if there is some theory.
\item Pictures! Really, even if you feel silly doing it or that you think you can’t draw. A picture can help tremendously in communicating difficult ideas.
\item Original examples over original data sets.
\item A conclusion telling us about future work and summarizing (again) the strong points of the paper.

\end{itemize}

\section{Text}
\begin{itemize}

\item Make text structure / ToC apparent in text (use transition words \& phrases). Try to keep a flow in your text, i.e., do not jump from one topic to the next without proper transitions.
\item No empty transition sections! If a section has subsections, add text before the first subsection to give an overview of what you will be discussing in the following subsections.
\item Try to motivate every step you are explaining it!
\item Have you run a spell check? No typos, please!
\item Has somebody read it for grammar mistakes? Have you used a grammar checker?
\item Does every sentence have a crisp statement? Have you been too wordy?
\item Does every statement have a citation?
\item Does a noun follow on every "This" like in "This NOUN is right there..."? Are there proper citations after every statement?
\item Keep sentences short! Have you applied Occams Razor to every sentence? Be to the point = concise!
\item Are you consistent with the tense?
\item Do you state all relevant parameters used for your experiments?
\item Use commas or dashes (--) everywhere where you want the reader to stop briefly and think.
\item Does one of your sentences clearly embody read exactly like one of your thoughts or ideas? If yes, consider making it into two sentences. It usually takes two sentences to convey one idea to another person.
\item Be prepared to remove even the most beautiful of sentences if it makes the paper as such better!
\item Make sure you do not discuss a concept that you introduce properly only later on in the paper. It often happens when rearranging papers! Check the flow between paragraphs of text.
\item Do you have a reference for everything?
\item Do you have widows or orphans? \url{http://www.acm.org/sigs/publications/sigfaq#a19}

\end{itemize}

\section{Bullet Points, Description Lists, etc}
\begin{itemize}

\item (Before the first and) after the last bullet point is always a sentence --- it never ends a paragraph or a section!
\item Avoid bullet points but numbered or description lists are OK!

\end{itemize}

\section{Section Headings}
\begin{itemize}

\item Are you using consistently British or American capitalization? (“Our Methodology” versus “Our algorithm”)
\item Are they conveying what we will find in the sections?

\end{itemize}

\section{Figures}
\begin{itemize}

\item Do you have at least one figure?
\item Does your robotics paper have a robot on its front page?
\item Do the figures look nice? Are the fonts in your figures large enough for easy browsing? Are the figures readable once printed out in black-and-white? Can we see any compression artifacts? Prefer vector graphics for your figures. Avoid screen shots unless absolutely necessary.
\item An interesting picture on the first page can be just as informative about the contents of the paper as the abstract.
\item Do all graphs have standard deviation bars?
\item Do you refer to all figures in the text?
\item Have you used the right font and font size in the figures? No [h] parameters in your text.
\item Try to have figures at the top of pages and spaced out over multiple pages.
\item For small figures, use wrapfig!
\item No Pixeled Graphics unless its a photo! Use PDF or EPS not JPEG, GIF, or BMP!
\item Do you have the correct ordering of your graphs according to their reference in the text?
\item The captions of the figures always need to explain the figure --- use a lot of text. Do not rely on the reference of in your text to explain a figure. No white-space around figures -- use wrapfig!

\end{itemize}

\section{Good Equations}
\begin{itemize}

\item No Fractions \$\textbackslash frac\{a\}\{b\}\$ in the text but always \$a/b\$.
\item Do not use large symbols (e.g. sum) in equations in the text inline!
\item Mention variables by name and parameter; e.g., " the width w and height h".
\item Explain ALL variables and all equations.
\item Equations, even if in a separate line, are embedded in the text. Thus, proper punctuation ("," or ".") should always be used.
\item Only use numbers for relevant equations
\item Do not use "*" for multiplication.
\item For vectors and matrices, always use bold font.
\item Do not use small brackets for large symbols, e.g. ( \textbackslash sum blabla ) but \textbackslash left ( \textbackslash sum blabla \textbackslash right)
\item Use "\textbackslash mathrm" for all mathematical symbols which are not known by latex. Do not use standard math-font! log reads as l \* o \* g and not as log.
\item After an equation is always a sentence --- it never ends a paragraph or a section!
\item Never ever use x * y as this means convolution \( \int{ x(t) y(t-\tau), d\tau}\). Never use \textbackslash cdot -- we are not in high school.

\end{itemize}

\section{Algorithms}
* Use a proper package and pseudo-code!

\begin{itemize}

\item Do you have a step-by-step toy example for every new algorithm being introduced? Present your examples early.

\end{itemize}

\section{Pages and page limits}
\begin{itemize}

\item Never exceed the page limit!
\item If the page limit is x pages, do you have an x pages long paper? Some reviewers feel you should use all the pages you
were granted.

\end{itemize}

\section{Format}
\begin{itemize}

\item Are references and citations formatted properly (e.g. look at Bibtex errors)?
\item If you refer to equations or section with a number (e.g., Section 4 ), always use upper case.
\item References to Equations should always come in brackets, e.g., Equation (4).
\item Use consistent capitalization for your headings (British vs American = German)

\end{itemize}
